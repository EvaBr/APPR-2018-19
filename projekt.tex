\documentclass[]{article}
\usepackage{lmodern}
\usepackage{amssymb,amsmath}
\usepackage{ifxetex,ifluatex}
\usepackage{fixltx2e} % provides \textsubscript
\ifnum 0\ifxetex 1\fi\ifluatex 1\fi=0 % if pdftex
  \usepackage[T1]{fontenc}
  \usepackage[utf8]{inputenc}
\else % if luatex or xelatex
  \ifxetex
    \usepackage{mathspec}
  \else
    \usepackage{fontspec}
  \fi
  \defaultfontfeatures{Ligatures=TeX,Scale=MatchLowercase}
\fi
% use upquote if available, for straight quotes in verbatim environments
\IfFileExists{upquote.sty}{\usepackage{upquote}}{}
% use microtype if available
\IfFileExists{microtype.sty}{%
\usepackage{microtype}
\UseMicrotypeSet[protrusion]{basicmath} % disable protrusion for tt fonts
}{}
\usepackage[margin=1in]{geometry}
\usepackage{hyperref}
\hypersetup{unicode=true,
            pdftitle={Poročilo pri predmetu Analiza podatkov s programom R},
            pdfauthor={Tjaša Renko},
            pdfborder={0 0 0},
            breaklinks=true}
\urlstyle{same}  % don't use monospace font for urls
\usepackage{graphicx,grffile}
\makeatletter
\def\maxwidth{\ifdim\Gin@nat@width>\linewidth\linewidth\else\Gin@nat@width\fi}
\def\maxheight{\ifdim\Gin@nat@height>\textheight\textheight\else\Gin@nat@height\fi}
\makeatother
% Scale images if necessary, so that they will not overflow the page
% margins by default, and it is still possible to overwrite the defaults
% using explicit options in \includegraphics[width, height, ...]{}
\setkeys{Gin}{width=\maxwidth,height=\maxheight,keepaspectratio}
\IfFileExists{parskip.sty}{%
\usepackage{parskip}
}{% else
\setlength{\parindent}{0pt}
\setlength{\parskip}{6pt plus 2pt minus 1pt}
}
\setlength{\emergencystretch}{3em}  % prevent overfull lines
\providecommand{\tightlist}{%
  \setlength{\itemsep}{0pt}\setlength{\parskip}{0pt}}
\setcounter{secnumdepth}{0}
% Redefines (sub)paragraphs to behave more like sections
\ifx\paragraph\undefined\else
\let\oldparagraph\paragraph
\renewcommand{\paragraph}[1]{\oldparagraph{#1}\mbox{}}
\fi
\ifx\subparagraph\undefined\else
\let\oldsubparagraph\subparagraph
\renewcommand{\subparagraph}[1]{\oldsubparagraph{#1}\mbox{}}
\fi

%%% Use protect on footnotes to avoid problems with footnotes in titles
\let\rmarkdownfootnote\footnote%
\def\footnote{\protect\rmarkdownfootnote}

%%% Change title format to be more compact
\usepackage{titling}

% Create subtitle command for use in maketitle
\newcommand{\subtitle}[1]{
  \posttitle{
    \begin{center}\large#1\end{center}
    }
}

\setlength{\droptitle}{-2em}

  \title{Poročilo pri predmetu Analiza podatkov s programom R}
    \pretitle{\vspace{\droptitle}\centering\huge}
  \posttitle{\par}
    \author{Tjaša Renko}
    \preauthor{\centering\large\emph}
  \postauthor{\par}
    \date{}
    \predate{}\postdate{}
  
\usepackage[slovene]{babel}
\usepackage{graphicx}

\begin{document}
\maketitle

\section{Izbira teme}\label{izbira-teme}

Za svojo projektno nalogo bom analizirala izpuste toplogrednih plinov po
Evropi. Primerjala bom razvitost držav (BDP) z okoljskimi davki in
letnimi izpusti.

\begin{center}\rule{0.5\linewidth}{\linethickness}\end{center}

\section{Obdelava, uvoz in čiščenje
podatkov}\label{obdelava-uvoz-in-ciscenje-podatkov}

Uvozila sem podatke o državah v obliki CSV in s pomočjo paketa Eurostat
iz spletne strani Eurostat. Podatki so podani v treh razpredelnicah.

\begin{enumerate}
\def\labelenumi{\arabic{enumi}.}
\tightlist
\item
  \texttt{Letni\ izpusti\ toplogrednih\ plinov} - podatki za posamezen
  plin, državo in leto
\end{enumerate}

\begin{itemize}
\tightlist
\item
  \texttt{LETO} - spremenljivka: leto, v katerem gledamo podatke
  (število),
\item
  \texttt{DRŽAVA} - spremenljivka: ime države (neurejen faktor),
\item
  \texttt{EMISIJE\ OGLJIKOVEGA\ DIOKSIDA\ V\ TONAH} - meritev: letni
  izpust v določeni državi (število),
\item
  \texttt{EMISIJE\ OGLJIKOVEGA\ DIOKSIDA\ V\ TONAH} - meritev: letni
  izpust v določeni državi (število),
\item
  \texttt{EMISIJE\ OGLJIKOVEGA\ DIOKSIDA\ V\ TONAH} - meritev: letni
  izpust v določeni državi (število).
\end{itemize}

\begin{enumerate}
\def\labelenumi{\arabic{enumi}.}
\setcounter{enumi}{1}
\tightlist
\item
  \texttt{Pregled\ držav} - splošni podatki, pripravljeni za primerjanje
\end{enumerate}

\begin{itemize}
\tightlist
\item
  \texttt{DRŽAVA} - spremenljivka: ime države(neurejen faktor),
\item
  \texttt{LETO} - spremenljivka: leto, v katerem gledamo podatke
  (število),
\item
  \texttt{ŠTEVILO\ PREBIVALCEV} - meritev: število prebivalcev določene
  države v določenem letu (število),
\item
  \texttt{GDP\ V\ MILIJONIH\ EVROV} - meritev: bruto domači proizvod
  določene države v določenem letu (število),
\item
  \texttt{GDP\ NA\ PREBIVALCA\ V\ EVRIH} - meritev: bruto domači
  proizvod na prebivalca določene države v določenem letu (število).
\end{itemize}

\begin{enumerate}
\def\labelenumi{\arabic{enumi}.}
\setcounter{enumi}{2}
\tightlist
\item
  \texttt{Okoljski\ davki} - pregled obdavčitev dejavnosti, ki slabo
  vplivajo na okolje
\end{enumerate}

\begin{itemize}
\tightlist
\item
  \texttt{DRŽAVA} - spremenljivka: ime države(neurejen faktor),
\item
  \texttt{LETO} - spremenljivka: leto, v katerem gledamo podatke
  (število),
\item
  \texttt{DAVKI\ NA\ ENERGIJO\ V\ MILIJONIH\ EVROV} - meritev:
  (število),
\item
  \texttt{DAVKI\ NA\ ONESNAŽEVANJE\ V\ MILIJONIH\ EVROV} - meritev:
  (število),
\item
  \texttt{DAVKI\ NA\ RABO\ NARAVNIH\ VIROV\ V\ MILIJONIH\ EVROV} -
  meritev: (število),
\item
  \texttt{DAVKI\ NA\ PROMET\ V\ MILIJONIH\ EVROV} - meritev: (število),
\item
  \texttt{OKOLJSKI\ DAVKI\ V\ MILIJONIH\ EVROV} - meritev: (število).
\end{itemize}

\begin{center}\rule{0.5\linewidth}{\linethickness}\end{center}

\section{Analiza in vizualizacija
podatkov}\label{analiza-in-vizualizacija-podatkov}

Prvi graf prikazuje letne emisije toplogrednih plinov v tonah in državah
med leti 2007 in 2016. Opazimo, da povečini letne emisije posamezne
države tekom danih let rahlo upadajo.

print(graf1)

Spodnji graf primerja GDP per habita z letnimi emisijami toplogrednih
plinov držav v tonah na prebivalca. Opazimo, da višji GDP ne pripomore k
manjšim emisijam; kvečjemu obratno.

print(graf2)

Naslednji graf prikazuje odstotek GDP Evropskih držav, namenjen
investicijam za varovanje okolja, med leti 2007 in 2013.

print(graf3)

Spodnji zemljevid prikazuje letne emisije vseh toplogrednih plinov
posameznih držav v tonah na prebivalca, merjene v letu 2013.

print(map)

Na tortnem diagramu so prikazani podatki o pobranih okoljskih davkih kot
deležu denarja, pobranega v ta namen, po celi Evropi.

print(tortni\_diagram)

\begin{center}\rule{0.5\linewidth}{\linethickness}\end{center}


\end{document}
